%--- Singe Page Web Application Framworks ------------------------------------------------------------------------------
\begin{frame}[fragile] \frametitle{Singe Page Web Application Framworks}
There are three dominating frameworks for the front end
\begin{itemize}
  \item React
  \item Vue
  \item Angular
\end{itemize}

React is the minimalistic approach
\\Angular aims for a complete package of features and tools
\\Vue is in the middle
\end{frame}

%--- Common Features ------------------------------------------------------------------------------
\begin{frame}[fragile] \frametitle{Common Features}

The application is built from a tree of components.
\\A component have:
\begin{itemize}
  \item a ''template`` for layout
  \item an object with:
  \begin{itemize} 
    \item properties for state
    \item functions for behaviour
  \end{itemize}
\end{itemize}

\vspace{5mm}
The framework:
\begin{itemize}
  \item maintain the application component tree (instantiates components)
  \item a mechanism for parent $\leftrightarrow$ child communication
  \item change detection: update the DOM when the application objects change
  \item life cycle hooks
\end{itemize}
\end{frame}

%--- Templates - react ------------------------------------------------------------------------------
\begin{frame}[fragile] \frametitle{Templates - React}

React:
\begin{itemize}
  \item \code{render()} --- a pure function generating a tree of react elements
  \item JSX makes it look like html
  \item but uses camelCase for some names
  \item assumption:
  \begin{itemize}
    \item Uppercase names are react components
    \item lower case names are html tags
  \end{itemize}
  \item JSX can embed JavaScript code
  \item conditions and loops are native JavaScript code
  \item {\tt babel} translates the templates to JavaScript code (ahead of time)
\end{itemize}
\end{frame}

%--- Templates - angular ------------------------------------------------------------------------------
\begin{frame}[fragile] \frametitle{Templates - Angular}
Angular:
\begin{itemize}
  \item is html (text in a separate file, or string template)
  \item components can have any legal XML tag name
  \item conditions and loops --- use \emph{directives}
  \item can embed ''TypeScript`` code
  \item context/namespace is the component object
  \item \emph{pipes} --- operation chanining
  \item \emph{directives} --- custom element attributes
  \item angular template compiler - just in time, or ahead of time compilation
\end{itemize}
\end{frame}

%--- Angular Template Syntax ------------------------------------------------------------------------------
\begin{frame}[fragile] \frametitle{Angular Template Syntax}
Data Binding:
\begin{itemize}
  \item interpolation \code{<div>\{\{} \emph{expression without side effects} \code{\}\}</div>}
  \item event binding: \code{<button (click)="deleteHero($event)">}
  \item bind DOM properties: \code{<img [src]="myImageUrl">}
  \item two way data binding: \code{<input [(ngModel)]="name">}
\end{itemize}
\end{frame}

%--- Angular Template Syntax ------------------------------------------------------------------------------
\begin{frame}[fragile] \frametitle{Angular Template Syntax}
Loops and conditions:
\begin{itemize}
  \item \code{<span *ngIf="value>10">}
  \item \code{<li *ngFor="let value of myList">}
  \item \code{<ul [ngSwitch]="me.framework">}\\
   \hspace{3mm} \code{<li *ngSwitchCase="angular">angular</li>}\\
   \hspace{3mm} \code{<li *ngSwitchDefault>html</li>}
\end{itemize}
\end{frame}

%--- Directives------------------------------------------------------------------------------
\begin{frame}[fragile] \frametitle{Directives}
Angular
\begin{itemize}
  \item looks like attributes in the template
  \item functions that have a reference to the DOM
  \begin{itemize}
    \item attribute directives
    \item structural directives
  \end{itemize}
  \item can be parametrised
  \item you can define own directives
\end{itemize}
\end{frame}

%--- Pipes------------------------------------------------------------------------------
\begin{frame}[fragile] \frametitle{Pipes}
Angular
\begin{itemize}
  \item functions that can be chained in the template
  \item angular comes with a stock of pipes:
  \begin{itemize}
    \item  date, uppercase, lowercase, currency, percent, json, async
  \end{itemize}
  \item can be parametrised
  \item you can define own pipes
\end{itemize}
\begin{CodeBox}{}
{{  birthday | date:'fullDate' | uppercase}}
{{  <li *ngFor="let item = http(url) | async"}}
\end{CodeBox}
\end{frame}

%--- Defining Components ------------------------------------------------------------------------------
\begin{frame}[fragile] \frametitle{Defining Components}
React
\begin{itemize}
  \item render function: \code{function MyComponent(props)\{ return ... \}}
  \item \code{class MyComponent extends React.Component}\\
    \hspace{3mm} \code{render() \{ return ... \}}
\end{itemize}
\vspace{1mm}
Angular
\begin{CodeBox}{child}

@Component({
  selector: 'app-my-component',
  templateUrl: './my.component.html',
  styleUrls: ['./my.component.css']
})
export class MyComponent { }
\end{CodeBox}
\end{frame}


%--- Parent-Child Communication ------------------------------------------------------------------------------
\begin{frame}[fragile] \frametitle{Parent-Child Communication}
React
\begin{itemize}
  \item no syntax for declaring an component interface
  \item parent $\rightarrow$ child:
  \begin{itemize}
    \item parent pass values as html attributes
    \item values are placed in the \code{props} object
  \end{itemize}
  \item child $\rightarrow$ parent
  \begin{itemize}
    \item parent must pass a \emph{call back function}
  \end{itemize}
\end{itemize}
\end{frame}

%--- Parent-Child Communication ------------------------------------------------------------------------------
\begin{frame}[fragile] \frametitle{Parent-Child Communication}
Angular\\
\begin{itemize}
  \item child declare interface: \code{@Input()} and \code{@Output()}
  \item paren uses data binding in template
  \item a parent can access a child object using \code{@ViewChild()}
\end{itemize}

\begin{CodeBox}{child}
export class MyComponent {
  @Input() item: string;
  @Output() myEvent = new EventEmitter<string>();
\end{CodeBox}
\begin{CodeBox}{parent}
<app-my-component
    [item]="currentItem"
    (myEvent)="addItem($event)">
</app-item-detail>
\end{CodeBox}
\end{frame}

%--- Change Detection ------------------------------------------------------------------------------
\begin{frame}[fragile] \frametitle{Change Detection}
React
\begin{itemize}
  \item the render function:
  \begin{itemize}
    \item must be a pure function of \code{this.state, this.props}
  \end{itemize}
  \item any state change must be done using \code{setState(newState)}
\end{itemize}
\vspace{5mm}
Angular
\begin{itemize}
  \item must use data bindings in the template
  \item you do not need to tell angular when the state changes
  \item angular detects any change in your objects and updates the DOM
\end{itemize}
\end{frame}

%--- Services ------------------------------------------------------------------------------
\begin{frame}[fragile] \frametitle{Services}
React
\begin{itemize}
  \item not supported: use the global name space
\end{itemize}
\vspace{5mm}
Angular
\begin{itemize}
  \item \emph{Dependency Injection}: parameters in the component constructor
  \item singleton objects accessible by any component
  \item service: a TypeScript class
\end{itemize}
\begin{CodeBox}{}
@Injectable({
  providedIn: 'root',
})
export class MyService {
}
\end{CodeBox}
\end{frame}

%--- Modules ------------------------------------------------------------------------------
\begin{frame}[fragile] \frametitle{Modules}
React
\begin{itemize}
  \item use JavaScript modules
\end{itemize}

\vspace{5mm}
Angular
\begin{itemize}
  \item encapsulates components, directives, pipes, and services
  \item exports the public parts
  \item can be lazy loaded
  \item modules can import other modules
  \item an application is always a \emph{root module}
\end{itemize}
\end{frame}

%--- Modules 2 -----------------------------------------------------------------------------
\begin{frame}[fragile] \frametitle{Modules}
\begin{CodeBox}{}
@NgModule({
  declarations: [
    AppComponent,
    ChildComponent
  ],
  imports: [
    BrowserModule,
    AppRoutingModule,
    FormsModule
  ],
  providers: [],
  exports:      [ ChildComponent ]
  bootstrap: [AppComponent]
})
export class AppModule { }

\end{CodeBox}
\end{frame}

%--- Routing ------------------------------------------------------------------------------
\begin{frame}[fragile] \frametitle{Routing in Angular}
\begin{CodeBox}{}
const appRoutes: Routes = [
  { path: 'compose-salad', component: ComposeSaladComponent },
  { path: '',
    redirectTo: '/compose-salad',
    pathMatch: 'full'
  },
  { path: '**', component: PageNotFoundComponent }
];
@NgModule({
  imports: [
    RouterModule.forRoot(appRoutes, { enableTracing: true })
  ],
})
\end{CodeBox}
\end{frame}

%--- Routing 2 ------------------------------------------------------------------------------
\begin{frame}[fragile] \frametitle{Routing in Angular}
\begin{CodeBox}{}
<nav>
  <a routerLink="/compose-salad" routerLinkActive="active">
    Komponera din egen sallad
  </a>
  <a routerLink="/orders" routerLinkActive="active">
    Order
  </a>
</nav>

<router-outlet></router-outlet>
<!-- Routed components go here -->
\end{CodeBox}
\end{frame}

%--- Command Line Tools------------------------------------------------------------------------------
\begin{frame}[fragile] \frametitle{Command Line Tools}
React
\begin{itemize}
  \item \code{npx create-react-app my-app}
  \item \code{rct gc <ComponentName>}
  \item \code{npm serve}
\end{itemize}
\vspace{5mm}
Angular
\begin{itemize}
  \item \tt{ng new my-app}
  \item \code{ng generate <schematic>}
  \item \code{ng build --prod}
  \item \code{ng serve --watch}
\end{itemize}
\end{frame}

%--- Libraries------------------------------------------------------------------------------
\begin{frame}[fragile] \frametitle{Libraries}
React
\begin{itemize}
  \item many from 3rd party suppliers
\end{itemize}
\vspace{5mm}
Angular
\begin{itemize}
  \item many built in
  \begin{itemize}
    \item http
    \item routing
    \item i18n
    \item reactive forms, template forms
  \end{itemize}
  \item many from 3rd party suppliers
\end{itemize}
\end{frame}


