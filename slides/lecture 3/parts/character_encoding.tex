\section{Character encoding}

%---------------------------------------------------------------------------------
\begin{frame}
\frametitle{Locales and Word Order}
\color{structure}
Text communicate infomation to the user. To handle text in a program you need:
\begin{itemize}
  \item encoding --- A mapping (value $ \leftrightarrow $ symbol)
  \item locale --- How to render dates, digits and time depends on where you are:
  \begin{itemize}
    \item Digits: 3.142 or 3,142?
    \item Date: 01/02/03
    \begin{itemize}\color{structure}
      \item 3 februari 2001?
      \item January 2, 2003?
      \item 1 February 2003?
    \end{itemize}
  \end{itemize}
  \item collation --- character order. Is Andersson before or after Åkesson?
\end{itemize}
\end{frame}

%---------------------------------------------------------------------------------
\begin{frame}[fragile]
\frametitle{Character encoding}
\color{structure}
There exists many different ways to encode characters
\begin{itemize}\color{structure}
  \item fixed width
  \item variable width (compare to Hoffman coding)
\end{itemize}
Some common standards:
\begin{itemize}\color{structure}
  \item Unicode and utf8, no just encoding, also collation (sorting)
  \item ISO-8859-1/latin 1
  \item UTF8 is conquering the world, it is standard for Java och JavaScript.
\end{itemize}
\end{frame}

%---------------------------------------------------------------------------------
\begin{frame}[fragile]
\frametitle{Unicode}
\color{structure}
A standard including:
\begin{itemize}\color{structure}
  \item visual reference
  \item set of standard character encodings
  \item an encoding method
  \item character properties (lower/upper case)
  \item rules for normalization, decomposition, collation
  \item rules for rendering, and bidirectional text display order (right-to-left, left-to-right scripts)
\end{itemize}
\end{frame}

\begin{frame}
\frametitle{Unicode Blocks (Simplified)}
\color{structure}
\begin{footnotesize}
\begin{table}
\begin{center}
\begin{tabular}{llll}
\hline
\textbf{Code}& \textbf{Name}& \textbf{Code}& \textbf{Name} \\
\hline
U+0000& Basic Latin& U+1400& Unified Canadian Aboriginal Syllabic \\
%\hline
U+0080& Latin-1 Supplement& U+1680& Ogham, Runic \\
%\hline
U+0100& Latin Extended-A& U+1780& Khmer \\
%\hline
U+0180& Latin Extended-B& U+1800& Mongolian \\
%\hline
U+0250& IPA Extensions& U+1E00& Latin Extended Additional \\
%\hline
U+02B0& Spacing Modifier Letters& U+1F00& Extended Greek  \\
%\hline
U+0300& Combining Diacritical Marks& U+2000& Symbols \\
%\hline
U+0370& Greek& U+2800& Braille Patterns \\
%\hline
U+0400& Cyrillic& U+2E80& CJK Radicals Supplement \\
%\hline
U+0530& Armenian& U+2F80& KangXi Radicals \\
%\hline
U+0590& Hebrew& U+3000& CJK Symbols and Punctuation \\
%\hline
U+0600& Arabic& U+3040& Hiragana, Katakana \\
%\hline
U+0700& Syriac& U+3100& Bopomofo \\
%\hline
U+0780& Thaana& U+3130& Hangul Compatibility Jamo \\
\hline
\end{tabular}
\end{center}
\end{table}
\end{footnotesize}
\end{frame}

\begin{frame}
\frametitle{Unicode Blocks (Simplified) (II)}
\color{structure}
\begin{footnotesize}
\begin{table}
\begin{center}
\begin{tabular}{llll}
\hline
\textbf{Code}& \textbf{Name}& \textbf{Code}& \textbf{Name} \\
\hline
%\hline
U+0900& Devanagari, Bengali& U+3190& Kanbun \\
%\hline
U+0A00& Gurmukhi, Gujarati& U+31A0& Bopomofo Extended \\
%\hline
U+0B00& Oriya, Tamil& U+3200& Enclosed CJK Letters and Months \\
%\hline
U+0C00& Telugu, Kannada& U+3300& CJK Compatibility \\
%\hline
U+0D00& Malayalam, Sinhala& U+3400& CJK Unified Ideographs Extension A \\
%\hline
U+0E00& Thai, Lao& U+4E00& CJK Unified Ideographs \\
%\hline
U+0F00& Tibetan& U+A000& Yi Syllables \\
%\hline
U+1000& Myanmar& U+A490& Yi Radicals \\
%\hline
U+10A0& Georgian& U+AC00& Hangul Syllables \\
%\hline
U+1100& Hangul Jamo& U+D800& Surrogates \\
%\hline
U+1200& Ethiopic& U+E000& Private Use \\
%\hline
U+13A0& Cherokee& U+F900& Others \\
\hline
\end{tabular}
\end{center}
\end{table}
\end{footnotesize}
\end{frame}

\begin{frame}[fragile]
\frametitle{The Unicode Encoding Schemes}
\color{structure}
\begin{itemize}
\item each character have a unique unicode
\item different ways to store the unique unicodes in a file:
  \begin{itemize}
    \item UTF-8, UTF-16, and UTF-32.\\
  \end{itemize}
\item UTF-16 used to be standard\\
\item uses 16 bits per character – 2 bytes –\\
\item \textit{FÊTE} \verb=0046 00CA 0054 0045=\\
\item UTF-8 has variable length for each character\\
\end{itemize}
\end{frame}

\begin{frame}
\frametitle{UTF-8}
\color{structure}
\begin{center}
\begin{tabular}{rl}
\hline
\multicolumn{1}{l}{\textbf{Range}}& \textbf{Encoding} \\
\hline
U-0000 -- U-007F& 0xxxxxxx \\
%\hline
U-0080 -- U-07FF& 110xxxxx 10xxxxxx \\
%\hline
U-0800 -- U-FFFF& 1110xxxx 10xxxxxx 10xxxxxx \\
%\hline
U-010000 -- U-10FFFF& 11110xxx 10xxxxxx 10xxxxxx 10xxxxxx \\
\hline
\end{tabular}
\end{center}
\end{frame}
